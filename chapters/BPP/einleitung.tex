\chapter{Projektbeschreibung}
\label{ch:description}
Der Titel des Projekts lautet \enquote{Automatische DNS-Zonen und Eintragsverwaltung in einer verteilten Multi-Cloud-Microservice-Architektur} und wird im Folgenden beschrieben.
Diese Arbeit soll sich auf das Kernthema der automatischen DNS-Zonen und Eintragsverwaltung in einer verteilten Multi-Cloud-Microservice-Architektur konzentrieren und nicht auf die Implementierung der Microservices selbst.
Weiterhin werden verwendete Technologien welche nicht direkt mit dem Kernthema in Verbindung stehen nur oberflächlich beschrieben.

\section{Idee}
\label{sec:description:projektidee}
Die Projektidee entstand aus dem Bedarf der Prozesstrennung und Automatisierung zwischen den Abteilungen der Entwicklung und dem Betrieb.
Die Abteilung des Betriebs besteht aus \ac{DevOps}, welche sich um die Entwicklung und Bereitstellung von Infrastruktur kümmern.
Die Abteilungen der Entwickler beschäftigen sich mit der Entwicklung von Microservices und der Bereitstellung dieser auf der Infrastruktur.

\medskip
\noindent

Die Laufzeitumgebung der Microservice-Container ist Kubernetes, welches auf den Cloud-Providern \ac{AWS}, \ac{GCP} und \ac{Azure} als verwalteter Service gebucht wird.
Kubernetes ist ein Container-Orchestrierungssystem, welches die Bereitstellung, Skalierung und Verwaltung von Containern ermöglicht und wird im Kapitel \ref{subsec:description:umfeld} genauer beschrieben.
Die Microservices werden in den Programmiersprachen Java, Python und Golang entwickelt und in Container-Images verpackt, was die enthaltenen Programmzeilen, Bibliotheken und Abhängigkeiten kapselt und somit eine effiziente Bereitstellung, Skalierung und Unveränderbarkeit ermöglicht.

\medskip
\noindent

Die implementierung mehrerer Cloud-Provicer ist notwendig, da die Microservices aufgrund von Kundenanforderungen auf verschiedenen Cloud-Providern bereitgestellt werden müssen.
Ein Kunde kann beispielsweise die Bereitstellung seiner Microservices auf \ac{AWS} fordern, während ein anderer Kunde die Bereitstellung auf \ac{GCP} oder \ac{Azure} fordert.
Die Wahl des Cloud-Providers ist für Kunden eine essenzielle Entscheidung, da Kunde komplexe Verträge mit den Cloud-Providern abschließen und diese nicht ohne weiteres wechseln können.
So ist das Produkt der SDA SE eine Plattform, welche die Bereitstellung von Microservices auf verschiedenen Cloud-Providern ermöglicht, eine simultane Bereitstellung auf mehreren Cloud-Providern ist möglich, jedoch im Standard nicht vorgesehen.


\subsection{Ziel}
\label{subsec:description:ziel}
blablabla

\subsection{Motivation}
\label{subsec:description:motivation}
blablabla

\section{Planung}
\label{sec:description:planung}
blablabla

\subsection{Umfeld}
\label{subsec:description:umfeld}
blablabla

\subsection{Plan}
\label{subsec:description:plan}
Ressourcenabschätzung, Teilaufgaben nennen, etc.

\subsection{Wirtschaftlichkeitsbetrachtung}
\label{subsec:description:wirtschaftlichkeitsbetrachtung}
blablabla

\subsection{Alternativen}
\label{subsec:description:alternativen}
(Literaturrecherche / Marktbetrachtung)

\section{Umsetzung}
\label{sec:description:umsetzung}
blablabla

\subsection{Organisation}
\label{subsec:description:Organisation}
blablabla

\subsection{Verlauf}
\label{subsec:description:verlauf}
blablabla

\subsection{Modifikation}
\label{subsec:description:modifikation}
blablabla

\section{Ergebnisse}
\label{sec:description:ergebnisse}
blablabla

\subsection{Darstellung}
\label{subsec:description:darstellung}
blablabla

\subsection{Analyse}
\label{subsec:description:analyse}
(Soll-/ Ist-Vergleich)

\subsection{Wirtschaftlichkeitsbetrachtung}
\label{subsec:description:ergebnisse:wirtschaftlichkeitsbetrachtung}
(Soll-/ Ist-Vergleich)

\section{Schlussfolgerung}
\label{sec:description:schlussfolgerung}
blablabla

\subsection{Auswirkungen}
\label{subsec:description:auswirkungen}
(Soll-/ Ist-Vergleich)

\subsection{Modifikationen}
\label{subsec:description:modifikationen}
(Soll-/ Ist-Vergleich)

\subsection{Bilanz}
\label{subsec:description:bilanz}
(Soll-/ Ist-Vergleich)
