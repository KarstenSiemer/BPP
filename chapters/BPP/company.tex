\chapter{Firma}
\label{ch:firma}
Die SDA SE ist ein Hamburger Plattform- und Ökosystementwickler.

%
% Section: Motivation
%
\section{Firmenchronik}
\label{sec:intro:firmenchronik}
Sie wurde 2016 von der SIGNAL IDUNA und der msg auf Initiative von Prof.
Dr. Markus Warg, Ideengeber der SDA und Leiter des Instituts für Service Design in Hamburg, gegründet.
%
% Section: Ziele
%
\section{Produktspekturm}
\label{sec:intro:produktspekturm}
Die SDA bietet eine digitale Service-Plattform, deren Alleinstellungsmerkmal die Service dominierte Architektur ist.
Diese wird im Netzwerk weltweit führender Servicewissenschaftler stetig weiterentwickelt.
Die Grundprinzipien der Offenheit und der Kooperation ermöglichen es Unternehmen, sich über ihr eigenes digitales Ökosystem mit Partnern bzw. Start-ups zu verbinden.
Die in Open-Source Software realisierten Services werden allen Anforderungen der modernen, digitalen Welt gerecht.
%
% Section: Struktur der Arbeit
%
\section{Betriebswirtschaftliche Kennzahlen}
\label{sec:intro:betrwirtschaftliche-kennzahlen}
Aktuelle Kennzahlen sind nicht öffentlich verfügbar und können daher nicht angegeben werden.

\section{Organisationsstruktur}
\label{sec:intro:organisationsstruktur}
Die SDA SE ist ein Gemeinschaftsunternehmen von SIGNAL IDUNA, msg, Allianz X, Debeka, und HUK-COBURG.
Dem Aufsichtsrat gehören Prof. Dr. Markus Warg, Daniela Rode, Vorstand der SIGNAL IDUNA Gruppe, Carsten Middendorf, Head of Platforms & Acquisitions von Allianz X, Daniel Thomas, Vorstand der HUK- COBURG, Dr. Normann Pankratz, Vorstand der Debeka Versicherungsgruppe sowie Dr. Jürgen Zehetmaier, Vorstand msg systems, an.
Den heutigen Vorstand bildet Marco Ziegler.

\section{Persönliche Aufgabenstellung}
\label{sec:intro:persoenliche-aufgabenstellung}
Zu den Aufgaben von Herrn Siemer gehören insbesondere:
• Multicloud Kubernetes Strategie:
o Ausarbeitung und Implementierung für AWS, GCP, und Azure
o Effiziente Container-Orchestrierung und -Verwaltung über diverse Cloud-Plattformen
• Security Strategie:
o Ausarbeitung und Implementierung für die gesamte Infrastruktur
o Anwendung von Kubernetes-spezifischen Sicherheitsmaßnahmen wie
Netzwerkrichtlinien, RBAC (Role-Based Access Control). Secrets Management
o AD management, Authorisierung und Authentisierung mittels verschiedener Flows wie
OAuth2
• Entwicklung automatisierter Prozesse für Continuous Integration und Continuous Deployment
• Programmierung von Cloud-Native Software mit Java, Python und Golang
• Erstellung von Microservice Container Images für effiziente Deployment und Skalierung
• Erweiterung des Cloud-Service-Angebots:
o Entwicklung und Optimierung von Cloud-Services
o Anpassung der Cloud-Infrastruktur an Unternehmensanforderungen
• Analyse und Behebung von Infrastrukturproblemen
• Sicherstellung der Systemverfügbarkeit und -leistung
• Implementierung von Monitoring- und Alerting-Systemen
• Zusammenarbeit und Schulung in DevOps-Praktiken
• Förderung einer Kollaborations- und Verbesserungskultur
