\chapter{Firma}
\label{ch:firma}
Die SDA SE ist ein in Hamburg ansässiges Unternehmen, das sich auf die Entwicklung von Plattformen und Ökosystemen spezialisiert hat.

\section{Firmenchronik}
\label{sec:intro:firmenchronik}
Die Gründung der SDA SE erfolgte 2016 auf Initiative von Prof. Dr. Markus Warg, der als Ideengeber der SDA und Leiter des Instituts
für Service Design in Hamburg, sowie Vorstand der SIGNAL IDUNA eine entscheidende Rolle spielte.
Die SDA SE wurde als Joint Venture zwischen der SIGNAL IDUNA, der MSG-Systems AG, ALLIANZ X und der IBM gegründet.
\medskip

\noindent
Diese wegweisenden Kooperationen führten zur Entstehung der SDA SE.
\medskip

\noindent
Im letzten Jahr wurde der ehemalige CEO Dr. Stephan Hans gegangen, was bis heute einige weitere interims CEO's und Umstrukturierungen zur Folge hatte.

\section{Ziele und Produkte}
\label{sec:intro:produktspekturm}
Die SDA SE bietet eine digitale Service-Plattform mit einer einzigartigen serviceorientierten Architektur.
Diese Plattform wird kontinuierlich von einem weltweiten Netzwerk führender Servicewissenschaftler weiterentwickelt.
\medskip

\noindent
Die Kernprinzipien sind Offenheit und Zusammenarbeit, die es Unternehmen ermöglichen, sich über eigene digitale Ökosysteme mit Partnern und Start-ups zu vernetzen.
Alle Services basieren auf Open-Source-Software und sind darauf ausgerichtet, die Anforderungen der modernen digitalen Welt zu erfüllen.

\section{Betriebswirtschaftliche Kennzahlen}
\label{sec:intro:betrwirtschaftliche-kennzahlen}
Leider können aktuelle betriebswirtschaftliche Kennzahlen aus vertraulichen Gründen nicht öffentlich zugänglich machen.
Ehrlich gesagt sind diese auch persönlich gar nicht bekannt.

\section{Organisationsstruktur}
\label{sec:intro:organisationsstruktur}
Die SDA SE ist ein Gemeinschaftsunternehmen von SIGNAL IDUNA, MSG, Allianz X, Debeka und HUK-COBURG.
Im Aufsichtsrat finden sich Persönlichkeiten wie Prof. Dr. Markus Warg, Daniela Rode (Vorstand der SIGNAL IDUNA Gruppe), Carsten Middendorf (Head of Platforms, Acquisitions bei Allianz X), Daniel Thomas (Vorstand der HUK-COBURG), Dr. Normann Pankratz (Vorstand der Debeka Versicherungsgruppe) sowie Dr. Jürgen Zehetmaier (Vorstand bei MSG-Systems).
\medskip

\noindent
Den heutigen Vorstand bildet Marco Ziegler.

\section{Persönliche Aufgabenstellung}
\label{sec:intro:persoenliche-aufgabenstellung}
Herr Siemer ist in seiner Position, als DevOps Engineer, für eine breite Palette von Aufgaben verantwortlich.
Er entwickelt und implementiert eine Multicloud-Kubernetes-Plattform, die AWS, GCP und Azure umfasst.
Diese Plattform wird von mehreren Teams und direkt von Kunden genutzt, um Microservices zu entwickeln und bereitzustellen.
\medskip

\noindent
Ziel ist es, eine effiziente Container-Orchestrierung und -Verwaltung über verschiedene Cloud-Plattformen sicherzustellen.
Zusätzlich gestaltet und implementiert er eine umfassende Sicherheitsstrategie für die gesamte Infrastruktur.
\medskip

\noindent
Dabei werden spezielle Kubernetes-Sicherheitsmaßnahmen wie Netzwerkrichtlinien, Role-Based Access Control (RBAC) und Secrets Management eingesetzt.
Er verantwortet auch das Management von Active Directory, die Autorisierung und Authentifizierung über verschiedene Flows wie OAuth2.
Herr Siemer arbeitet an der Entwicklung automatisierter Prozesse für Continuous Integration und Continuous Deployment und programmiert Cloud-native Software in den Sprachen Java, Python und Golang.
Darüber hinaus ist er für die Erstellung von Microservice-Container-Images verantwortlich, um eine effiziente Bereitstellung und Skalierung zu ermöglichen.
\medskip

\noindent
Er implementiert Monitoring- und Alerting-Systeme und arbeitet an der Förderung einer Kultur der Zusammenarbeit und kontinuierlichen Verbesserung im Sinne von DevOps-Praktiken.
Herr Siemer analysiert und behebt Infrastrukturprobleme und stellt die Systemverfügbarkeit und -leistung sicher.
Erweitern des Cloud-Service-Angebots, die Optimierung von Cloud-Services und die Anpassung der Cloud-Infrastruktur an die Anforderungen des Unternehmens sind weitere Aufgaben.
