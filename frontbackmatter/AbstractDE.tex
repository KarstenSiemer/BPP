%*******************************************************
% Abstract in German
%*******************************************************
\begin{otherlanguage}{ngerman}
	\pdfbookmark[0]{Zusammenfassung}{Zusammenfassung}
	\chapter*{Zusammenfassung}
	Diese Arbeit befasst sich mit der automatischen DNS-Zonen- und Eintragsverwaltung in einer verteilten Multi-Cloud-Microservice-Architektur.
	\medskip
	Das Ziel besteht darin, die Entwickler von manuellen Tätigkeiten zu entlasten und die Effizienz zu steigern.
	Die Arbeit umfasst die Auswahl und Implementierung eines Kubernetes Operators, sowie die Konzeption einer DNS-Hierarchie,
	die Sicherheit von öffentlichen und privaten Zonen, die Implementierung von DNSSEC und eine Infrastruktur-Übersetzung mittels Terraform.
	\medskip
	Der Projektverlauf folgt einem iterativen Modell, wobei wöchentliche Meetings die Organisation sicherstellen.
	\medskip
	Die Ergebnisse entsprechen den Anforderungen, wurden erfolgreich getestet und erreichen den Break-Even-Point nach vier Tagen, was die Wirtschaftlichkeit der Lösung unterstreicht.
	Die Auswirkungen auf die Entwickler sind positiv, da sie nun Microservices auf verschiedenen Cloud-Providern ohne manuelle DNS-Konfiguration bereitstellen können.
	\medskip
	Die Bilanz zeigt, dass die Automatisierung der DNS-Infrastruktur einen wichtigen Schritt in Richtung Effizienz und Sicherheit darstellt.
\end{otherlanguage}
